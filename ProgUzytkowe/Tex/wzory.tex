\documentclass{article}
\usepackage[a4paper,left=3.5cm,right=2.5cm,top=2.5cm,bottom=2.5cm]{geometry}
\usepackage[MeX]{polski}
\usepackage[cp1250]{inputenc}
%\usepackage{polski}
%\usepackage[utf8]{inputenc}
\usepackage[pdftex]{hyperref}
\usepackage{makeidx}
\usepackage[tableposition=top]{caption}
\usepackage{algorithmic}
\usepackage{graphicx}
\usepackage{enumerate}
\usepackage{multirow}
\usepackage{amsmath} %pakiet matematyczny
\usepackage{amssymb} %pakiet dodatkowych symboli
\usepackage{minted}
\begin{document}
Tu umieszczamy kod TeXa, ktory bedzie kompilowany, $\sum$. Wi�cej tekstu
\begin{displaymath}
	S^{C_{i}}(a)=\frac{(\overline{C}^{a}_{i}-\widehat{C}^{a}_{i})^2}{Z_{\overline{C}^{{a}^2}_{i}}+Z_{\widehat{C}^{{a}^2}_{i}}}, a \in A.
\end{displaymath}

\begin{equation}
\left[ 
\begin{array}{cccc}
a_{11} & a_{12} & \ldots & a_{1K} \\
a_{21} & a_{22} & \ldots & a_{1K} \\
\vdots & \vdots & \ddots & \vdots  \\
a_{K1} & a_{K2} & \ldots & a_{KK} \\
\end{array}
\right]*
\left[
\begin{array}{c}
x_1 \\
x_2 \\
\vdots\\
x_K \\
\end{array}
\right] =
\left[
\begin{array}{c}
b_1 \\
b_2 \\
\vdots\\
b_K \\
\end{array}
\right]
\end{equation}
\begin{verbatim}
for(int i=0;i<10;i++)
{
	cout<<"i="<<i;
}
\end{verbatim}

\begin{algorithmic}
\FOR{i=0,1,$\ldots$,10}
	\item{\verb+cout<<"i="<<i;+}
\ENDFOR
\end{algorithmic}
\begin{minted}{c}
for(int i=0;i<10;i++)
{
	printf("Hello world");
	return 0;
}
\end{minted}


\end{document}