\documentclass{article}
\usepackage[a4paper,left=3.5cm,right=2.5cm,top=2.5cm,bottom=2.5cm]{geometry}
\usepackage[MeX]{polski}
\usepackage[cp1250]{inputenc}
%\usepackage{polski}
%\usepackage[utf8]{inputenc}
\usepackage[pdftex]{hyperref}
\usepackage{makeidx}
\usepackage[tableposition=top]{caption}
\usepackage{algorithmic}
\usepackage{graphicx}
\usepackage{enumerate}
\usepackage{multirow}
\usepackage{amsmath} %pakiet matematyczny
\usepackage{amssymb} %pakiet dodatkowych symboli
\usepackage[table]{xcolor} %pakiet do kolorowania tabel
\begin{document}

\begin{table}
	\centering
	\rowcolors{2}{lightgray}{yellow}
		\begin{tabular}{|c|c|c|c|c|c|c|}
		\hline
		\multirow{2}{*}{No. of visual words} & \multicolumn{6}{c|}{Dataset} \\ \cline{2-7}
		& 1 & 2 & 3 & 4 & 5 & 6 \\ \hline 
	50 & 61.27\% & 88.92\% &77.88\%& 87.89\%& 92.04\% & 96.65\% \\ \hline
		\end{tabular}
	\caption{Final classifaction}
	\label{tab:Final}
\end{table}
Zadania do zrobienia: \\
 2 (Tabela 11), 3 (bez numeru), jedna kolorowa, 9 (Tabela 15)\\
Rysunki: \\
5, 6, 7
Doka�czanie komend: ctrl+spacja \\
Dodaj komentarz wielolinijkowy: ctrl+Q \\
Usu� komentarz wielolinijkowy: ctrl+W\\
\end{document}