\documentclass[]{beamer}
\usepackage[MeX]{polski}
\usepackage[cp1250]{inputenc}
%\usepackage{polski}
%\usepackage[utf8]{inputenc}
\beamersetaveragebackground{blue!10}
\usetheme{Warsaw}
\usecolortheme[rgb={0.1,0.5,0.7}]{structure}
\usepackage{beamerthemesplit}
\usepackage{multirow}
\usepackage{multicol}
\usepackage{array}
\usepackage{graphicx}
\usepackage{enumerate}
\usepackage{amsmath} %pakiet matematyczny
\usepackage{amssymb} %pakiet dodatkowych symboli

\title{Przykladowy pusty Beamer}
\date{}

\begin{document}

\frame
{
	\maketitle
}
\frame
{
	\frametitle{Przyk�ady u�ycia blok�w}
	\begin{block}
		{To jest zwyk�y blok}
		Tutaj umie�cimy jego tre��
	\end{block}
	\begin{exampleblock}
		{To jest przyk�ad}
		\texttt{for(int i=0;i<n;i++)cout<<"Hello world";}
	\end{exampleblock}
	\begin{alertblock}
		{Uwaga!}
		Tu wpisujemy tekst, kt�ry wymaga podkre�lenia.
	\end{alertblock}
}

\frame
{
\frametitle{Om�wione zagadnienia z Texa}
\begin{enumerate}
	\item <1-2>formatowanie tekstu i bibliografia
	\item <3-4>wzory matematyczne i pakiet algorithmic
	\item <2-3>tabele i obrazki
	\item <3-4>pakiet beamer do tworzenia prezentacji
\end{enumerate}
}



\end{document}