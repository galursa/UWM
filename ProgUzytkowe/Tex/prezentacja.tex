\documentclass[]{beamer}
\usepackage[MeX]{polski}
\usepackage[cp1250]{inputenc}
%\usepackage{polski}
%\usepackage[utf8]{inputenc}
\beamersetaveragebackground{blue!10}
\usetheme{Warsaw}
\usecolortheme[rgb={0.1,0.5,0.7}]{structure}
\usepackage{beamerthemesplit}
\usepackage{multirow}
\usepackage{multicol}
\usepackage{array}
\usepackage{graphicx}
\usepackage{enumerate}
\usepackage{amsmath} %pakiet matematyczny
\usepackage{amssymb} %pakiet dodatkowych symboli

\title{Prezentacja na temat \TeX a}
\date{}

\begin{document}

\frame
{
\maketitle
}
\frame
{
	\frametitle{Bloki w \TeX u}
	\begin{block}
	{Przyk�adowy blok w \TeX u}
	Tu wpisujemy tre��
	\end{block}
	\begin{exampleblock}
	{A to jest blok do przyk�ad�w}
	Na przyk�ad
	\end{exampleblock}
	\begin{alertblock}
	{Uwaga!}
	U�ywamy gdy chcemy podkre�li� co� wa�nego.
	\end{alertblock}
}
\frame
{
	\frametitle{Podpunkty i widoczno��}
	Zagadnienia z \TeX a, kt�re omawiali�my na �wiczeniach z program�w u�ytkowych
	\begin{itemize}
		\item<1-3> sk�adanie tekstu w \TeX u
		\item<2-3> wzory matematyczne i �rodowisko algorithmic
		\item<2-4> tabele, rysunki
		\item<3-4> pakiet beamer - do tworzenia prezentacji
	\end{itemize}
}
\frame
{
	\frametitle{wz�r matematyczny}
	\begin{displaymath}
		a^2+b+c-5=0
	\end{displaymath}
}
\end{document}